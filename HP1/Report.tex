\documentclass[18pt,a4paper]{article}
\usepackage[letterpaper]{geometry}
\geometry{verbose,tmargin=2cm,bmargin=2cm,lmargin=2.50cm,rmargin=2.50cm}
\usepackage[spanish]{babel}
\usepackage[utf8x]{inputenc}
%\usepackage[latin1]{inputenc}
\usepackage{amsmath,amsthm,amssymb,amsfonts,amstext}
\usepackage[pdftex]{graphicx}
\usepackage{caption}
\usepackage{subcaption}
\usepackage{framed}
\usepackage{wrapfig}
\DeclareGraphicsExtensions{.bmp,.png,.pdf,.jpg}
\setcounter{tocdepth}{3}
\usepackage{textcomp}
\usepackage{tikz}
\usepackage{circuitikz}
\usetikzlibrary{decorations.pathmorphing,patterns}
\usepackage{pgf}
\usepackage{multicol}
\usepackage{lipsum}
\usepackage{mathabx}
\usepackage{microtype}
\usepackage{vwcol}
\usepackage{pdflscape} % landscape specific pages
\usepackage{verbatim}
\usepackage{enumerate}
\usepackage{setspace}
\onehalfspacing
\decimalcomma
\deactivatequoting

\renewcommand{\contentsname}{Contenido} % Renombrar a Español
\renewcommand{\figurename}{Figura}
\renewcommand{\tablename}{Tabla}
\renewcommand{\refname}{Bibliografía}

\begin{document}

\begin{framed}
\begin{center}
%	\begin{multicols}{2}
%		\begin{minipage}{\textwidth}
%\begin{flushleft} 
{\bf \large {Stochastic optimization algorithms, 2016}}\\
		{\large\noindent\em\bf Report, Home Problems set1}\\
		Jose Esteban Pérez-Hidalgo, 880416T670

%\end{flushleft}
%\end{minipage}
	%	~
              %\includegraphics[width=0.2\textwidth]{/home/esteban/Dropbox/Public/image001.jpg}

  %      ~			
 %             %\includegraphics[width=0.2\textwidth]{/home/esteban/Dropbox/Public/image004.jpg}
		
%	\end{multicols}
\end{center}
\end{framed}

\section{Problem 1.1 Penalty Method}

\begin{enumerate}

	\item In this case we have only one inequality to include in the 
	penalty term. The penalty term is as follows: $\mu\, (x_1^2 + x_2^2 - 1)^2$.\\
	Then, the function $f_p({\bf x};\mu)$ takes the form: $$f_p({\bf x};\mu) = 
	(x_1-1)^2 + 2(x_2-2)^2 + \mu\, (x_1^2 + x_2^2 - 1)^2$$
	
	\item Now the gradient of ${\it f}_p({\bf x};\mu)$ is computed taking 
	the partial derivatives about the variables $x_1$ and $x_2$: 
	$$\nabla f_p({\bf x};\mu) = \Biggl(\,\frac{\partial f_p}{\partial 
	x_1},\frac{\partial f_p}{\partial x_2}\,\Biggr)^T$$
	
	Where:
	$$\dfrac{\partial f_{p}}{\partial x_1} = 2(x_1 - 1)\; ; \;\dfrac{\partial f_{p}}{\partial x_2} = 4(x_2 - 2)$$
	$$\Rightarrow \nabla f_p({\bf x};\mu) = \Bigl(\,2(x_1 - 1), 4(x_2 - 2)\,\Bigr)^T$$
	
	\item The unconstrained minimum of the function can be computed 
	equaling both partial derivatives found above, to zero:
	$$\dfrac{\partial f_{p}}{\partial x_1} = 2(x_1 - 1) = 0 \Rightarrow 
	x_1 = 1$$	
	$$\dfrac{\partial f_{p}}{\partial x_2} = 4(x_2 - 2) = 0 \Rightarrow 
	x_2 = 2$$
	
	Thus, the minimum of the uncostrained function, i.e., the starting 
	point of the penalty method is:	\begin{center}{\tt startingPoint} = 
	$(1,2)^T$ \end{center}
	
	\item The program functions required are located in the corresponding directory according to the instruccions given.
	
	\item Parameters used in {\tt PenaltyMethod.m}:
		
		\begin{table}[h]
			\centering
			%\caption{My caption}
			%\label{my-label}
			\begin{tabular}{lll}\hline
				$\mu$ 	& $x_1^{*}$ 		& $x_1^{*}$\\ \hline \hline
				1   	& 1.848         & 1.627 \\ %\hline
				10  	& 1.896         & 1.618 \\ %\hline
				100 	& 1.901         & 1.617 \\ %\hline
				400 	& 1.901         & 1.617 \\ \hline 
			\end{tabular}
		\end{table}
	
\end{enumerate}

\newpage

\section{Problem 1.3 Basic GA program}
	\begin{enumerate}[a)]

		\item The program functions required are located in the corresponding directory according to the instruccions given.
		
		\item For each variable 50 genes were used. The main function is {\tt FunctionOptimization.m} is does not need any extra manual parameters to execute.\\ Following is presented the table with the set of parameters used to get results for 20 runs. The last three columns correspond to the averaged results for $x_1^{*}$, $x_2^{*}$ and the Best Fit computed by the program.
			\begin{table}[h]
			%\resizebox{\textwidth}{!}
				\centering
				\begin{tabular}{l l l l l l l l l}\hline
				Set & Pop Size\footnote{Population Size} & Cross Prob\footnote{Crossover Probability} & Mut Prob\footnote{Mutation Probability} & Tour Size\footnote{Tournament Size} & Tour Selec Param\footnote{Tournament Selection Parameter} & $x_1^{*}$ & $x_2^{*}$ & Best Fit  \\ \hline
				$1^{st}$     & 50             & 0.8                  & 0.025                & 2               & 0.75                            & 0.18616 & -0.87919 & 0.29325 \\
				$2^{nd}$     & 100            & 0.5                  & 0.1                  & 4               & 0.5                             & 0.00029 & -0.99898 & 0.33169  \\
				$3^{rd}$     & 500            & 0.55                 & 0.02                 & 3               & 0.9                             & 0.0      & -1.0    & 0.3       \\
				$4^{th}$     & 1000           & 0.65                 & 0.3                  & 5               & 0.8                             & 0.00006 & -0.99805 & 0.32811 \\
				$5^{th}$     & 10000          & 0.9                  & 0.01                 & 10              & 0.85                            & 0.0      & -1.0      & 0.3\\ \hline      
				\end{tabular}
			\end{table}
		
		\item To probe that the point $(x_1^{*} = 0, x_2^{*} = -1)^T$ is actually an stationary point of the function, the point must satisfy the ecuations:
			\begin{equation} \label{eq:parcial1}
				\dfrac{\partial g(x_1^{*},x_2^{*})}{\partial x_1} = 0
			\end{equation}
			\begin{equation} \label{eq:parcial2}	
				\dfrac{\partial g(x_1^{*},x_2^{*})}{\partial x_2} = 0
			\end{equation}
			To compute de partial derivatives, and for simplicity, the function $g(x_1,x_2)$ will be expressed as $g = \cal X \cdot \cal Y$, where:
			\begin{equation}\label{eq:x}
				{\cal X} = 1 + f^{2}h
			\end{equation}
			with: $f = x_1 + x_2 + 1, \>\text{and}\> h = 19 - 14x_1 + 3x_1^2 - 14x_2 + 6x_1x_2 + 3x_2^2 $\\
			
			and:
			\begin{equation}\label{eq:y}
				{\cal Y} = 30 + a^2b
			\end{equation}
			with: $a = 2x_1 - 3x_2, \>\text{and}\> b = 18 - 32x_1 + 12x_1^2 + 48x_2 - 36x_1x_2 + 27x_2^2 $\\
			
			Now, the partial derivative of $g$ is 
			
			\begin{equation}\label{eq:dg}
				\dfrac{\partial g}{\partial x} = \dfrac{\partial{\cal X}}{\partial x} {\cal Y} + {\cal X}\dfrac{\partial{\cal Y}}{\partial x}
			\end{equation} 
			
			where $x = \{x_1,x_2\}$. In terms of $f$, $g$, $a$ and $b$, this derivatives take the form: 

			\begin{equation}\label{eq:dx}
				\dfrac{\partial{\cal X}}{\partial x} = 2f\dfrac{\partial f}{\partial x}h + f^2\dfrac{\partial h}{\partial x}
			\end{equation}
			
			\begin{equation}\label{eq:dy}
				\dfrac{\partial{\cal Y}}{\partial x} = 2a\dfrac{\partial a}{\partial x}b + a^2\dfrac{\partial b}{\partial x}
			\end{equation}
			Here one can note that both $\dfrac{\partial{\cal X}}{\partial x}$ and $\cal X$ contain the function $f$, this is covenient because if the point in question is evaluated in $f$ it results that: $f(x_1^{*},x_2^{*}) = 0$. Then, replacing this result in equations \eqref{eq:x} and \eqref{eq:dx}, both partial derivatives of $\cal X$ vanish, and the function $\cal X$ itself is equal to 1.
			
			Thus, \eqref{eq:dg} simplifies to:
			\begin{equation}\label{eq:dg2}
				\dfrac{\partial g}{\partial x_1} = \dfrac{\partial{\cal Y}}{\partial x_1},\> \dfrac{\partial g}{\partial x_2} = \dfrac{\partial{\cal Y}}{\partial x_2}
			\end{equation} 
			
			In order to compute $\dfrac{\partial{\cal Y}}{\partial x_1}$ and $\dfrac{\partial{\cal Y}}{\partial x_2}$, first, is needed to find: $\dfrac{\partial{a}}{\partial x_1}$, $\dfrac{\partial{a}}{\partial x_2}$, $\dfrac{\partial{b}}{\partial x_1}$ and $\dfrac{\partial{b}}{\partial x_2}$. The calculation is simple:
			
			\begin{equation}\label{eq:ax1}
				\dfrac{\partial{a}}{\partial x_1} = 2
			\end{equation} 
	
			\begin{equation}\label{eq:ax2}
				\dfrac{\partial{a}}{\partial x_2} = -3
			\end{equation} 
			
			\begin{equation}\label{eq:bx1}
				\dfrac{\partial{b}}{\partial x_1} = -32 + 24x_1 - 36x_2
			\end{equation} 
			
			\begin{equation}\label{eq:bx1}
				\dfrac{\partial{b}}{\partial x_2} = 48 - 36x_1 + 54x_2
			\end{equation} 
			
			The last step is to evaluate the point in question in: $a$, $b$, $\dfrac{\partial{b}}{\partial x_1}$, and $\dfrac{\partial{b}}{\partial x_2}$, which results in:
			
			\begin{equation}\label{eq:ae}
				a(x_1^{*},x_2^{*}) = 3
			\end{equation}

			\begin{equation}\label{eq:be}
				b(x_1^{*},x_2^{*}) = -3
			\end{equation}
			
			\begin{equation}\label{eq:bx1e}
				\dfrac{\partial{b(x_1^{*},x_2^{*})}}{\partial x_1} = 4
			\end{equation}
			
			\begin{equation}\label{eq:bx2e}
				\dfrac{\partial{b(x_1^{*},x_2^{*})}}{\partial x_2} = -6
			\end{equation}
			
			Finally, replacing equations \eqref{eq:ax1}, \eqref{eq:ax2} and \eqref{eq:ae} to \eqref{eq:bx2e} into \eqref{eq:dy} for $x = \{x_1, x_2\}$, both equations \eqref{eq:dg2} vanish.
			Hence, the point $(x_1^{*} = 0, x_2^{*} = -1)^T$ satisfies equations \eqref{eq:parcial1} and \eqref{eq:parcial2}, meaning that the point is indeed a stationary point of $g$.  	
	\end{enumerate}

\end{document}
